%% beamer/knitr slides
%% for Statistical Modeling and Data Visualization course @ UMass
%% Nicholas Reich: nick [at] schoolph.umass.edu


\documentclass[table]{beamer}\usepackage[]{graphicx}\usepackage[]{color}
% maxwidth is the original width if it is less than linewidth
% otherwise use linewidth (to make sure the graphics do not exceed the margin)
\makeatletter
\def\maxwidth{ %
  \ifdim\Gin@nat@width>\linewidth
    \linewidth
  \else
    \Gin@nat@width
  \fi
}
\makeatother

\definecolor{fgcolor}{rgb}{0.345, 0.345, 0.345}
\newcommand{\hlnum}[1]{\textcolor[rgb]{0.686,0.059,0.569}{#1}}%
\newcommand{\hlstr}[1]{\textcolor[rgb]{0.192,0.494,0.8}{#1}}%
\newcommand{\hlcom}[1]{\textcolor[rgb]{0.678,0.584,0.686}{\textit{#1}}}%
\newcommand{\hlopt}[1]{\textcolor[rgb]{0,0,0}{#1}}%
\newcommand{\hlstd}[1]{\textcolor[rgb]{0.345,0.345,0.345}{#1}}%
\newcommand{\hlkwa}[1]{\textcolor[rgb]{0.161,0.373,0.58}{\textbf{#1}}}%
\newcommand{\hlkwb}[1]{\textcolor[rgb]{0.69,0.353,0.396}{#1}}%
\newcommand{\hlkwc}[1]{\textcolor[rgb]{0.333,0.667,0.333}{#1}}%
\newcommand{\hlkwd}[1]{\textcolor[rgb]{0.737,0.353,0.396}{\textbf{#1}}}%
\let\hlipl\hlkwb

\usepackage{framed}
\makeatletter
\newenvironment{kframe}{%
 \def\at@end@of@kframe{}%
 \ifinner\ifhmode%
  \def\at@end@of@kframe{\end{minipage}}%
  \begin{minipage}{\columnwidth}%
 \fi\fi%
 \def\FrameCommand##1{\hskip\@totalleftmargin \hskip-\fboxsep
 \colorbox{shadecolor}{##1}\hskip-\fboxsep
     % There is no \\@totalrightmargin, so:
     \hskip-\linewidth \hskip-\@totalleftmargin \hskip\columnwidth}%
 \MakeFramed {\advance\hsize-\width
   \@totalleftmargin\z@ \linewidth\hsize
   \@setminipage}}%
 {\par\unskip\endMakeFramed%
 \at@end@of@kframe}
\makeatother

\definecolor{shadecolor}{rgb}{.97, .97, .97}
\definecolor{messagecolor}{rgb}{0, 0, 0}
\definecolor{warningcolor}{rgb}{1, 0, 1}
\definecolor{errorcolor}{rgb}{1, 0, 0}
\newenvironment{knitrout}{}{} % an empty environment to be redefined in TeX

\usepackage{alltt}


\input{../../slide-includes/standard-knitr-beamer-preamble}

%	The following variables are assumed by the standard preamble:
%	Global variable containing module name:
\title{Introduction to Data Visualization}
%	Global variable containing module shortname:
%		(Currently unused, may be used in future.)
\newcommand{\ModuleShortname}{introRegression}
%	Global variable containing author name:
\author{Nicholas G Reich}
%	Global variable containing text of license terms:
\newcommand{\LicenseText}{Made available under the Creative Commons Attribution-ShareAlike 3.0 Unported License: http://creativecommons.org/licenses/by-sa/3.0/deed.en\textunderscore US }
%	Instructor: optional, can leave blank.
%		Recommended format: {Instructor: Jane Doe}
\newcommand{\Instructor}{}
%	Course: optional, can leave blank.
%		Recommended format: {Course: Biostatistics 101}
\newcommand{\Course}{}


\input{../../slide-includes/shortcuts}

\hypersetup{colorlinks,linkcolor=,urlcolor=MainColor}


%	******	Document body begins here	**********************
\IfFileExists{upquote.sty}{\usepackage{upquote}}{}
\begin{document}

%	Title page
\begin{frame}[plain]
	\titlepage
\end{frame}

%	******	Everything through the above line must be placed at
%		the top of any TeX file using the statsTeachR standard
%		beamer preamble.


%%%%%%%%%%%%%%%%%%%%%%%%%%%%%%%%%%%%%%%%%%
% KEY POINTS/OUTLINE
  % - show a few data visualizations, in pairs critique them (2 things you like, 2 you don't), then discuss as a class
  % - why do we viz our data? exploratory, hypothesis generating, describe results, summarize models, show patterns
  % - key principles
  % - lessons about how we perceive features of data graphics
  % - Exploratory graphics in R: demo code on NHANES data?
  % - more pairs critiques, including 1-2 of the same from earlier

%%%%%%%%%%%%%%%%%%%%%%%%%%%%%%%%%%%%%%%%%%


\begin{frame}{Key principles of effective data graphics}

\begin{itemize}
    \item Know your audience
    \item ``{\bf Show} the data''
    \item ``Encourage the eye to {\bf compare} different pieces of data''
    \item {\bf Simplify} by maximizing the ``data-ink ratio.''
    \item Leverage color, shapes, facets to highlight multivariate data.
    \item Annotate your figures with context.
\end{itemize}

\end{frame}


%%%%%%%%%%%%%%%%%%%%%%%%%%%%%%%%%%%%%%%%%%

\begin{frame}[fragile]{Visual cues}
\begin{block}{Graphical elements that draw attention}

The choice of which visual cues you use inform the story that you are able to convey and the points you can highlight.

\end{block}

\includegraphics[width=\textwidth]{figure-static/visual-cues.png}

\end{frame}

%%%%%%%%%%%%%%%%%%%%%%%%%%%%%%%%%%%%%%%%%%

\begin{frame}[fragile]{Visual cues: position (numerical)}

\begin{block}{Where are the data in relation to each other?}

e.g. points and axis alignment.
\end{block}

\vspace{15em}

\end{frame}

%%%%%%%%%%%%%%%%%%%%%%%%%%%%%%%%%%%%%%%%%%

\begin{frame}[fragile]{Visual cues: Length (numerical)}

\begin{block}{How big (in one dimension)?}

e.g. bars, lines (aligned), lines(unaligned) ...

\end{block}

\vspace{15em}

\end{frame}


%%%%%%%%%%%%%%%%%%%%%%%%%%%%%%%%%%%%%%%%%%

\begin{frame}[fragile]{Visual cues: Angle (numerical)}

\begin{block}{How wide? Parallel to something else?}

e.g. lines, pie wedges, ...

\end{block}

\vspace{15em}

\end{frame}

%%%%%%%%%%%%%%%%%%%%%%%%%%%%%%%%%%%%%%%%%%

\begin{frame}[fragile]{Visual cues: Direction/slope (numerical)}

\begin{block}{Up or down? At what slope?}

e.g. lines, time-series, ...

\end{block}

\vspace{15em}

\end{frame}

%%%%%%%%%%%%%%%%%%%%%%%%%%%%%%%%%%%%%%%%%%

\begin{frame}[fragile]{Visual cues: Shape (categorical)}

\begin{block}{Belonging to which group?}

e.g. points

\end{block}

\vspace{15em}

\end{frame}

%%%%%%%%%%%%%%%%%%%%%%%%%%%%%%%%%%%%%%%%%%

\begin{frame}[fragile]{Visual cues: Area/volume (numerical)}

\begin{block}{How big (in 2/3 dimensions)?}

e.g. circles, squares

\end{block}

\vspace{15em}

\end{frame}


%%%%%%%%%%%%%%%%%%%%%%%%%%%%%%%%%%%%%%%%%%

\begin{frame}[fragile]{Visual cues: Shade/intensity (categorical or numerical)}

\begin{block}{To what extent? How severely?}

e.g. points, lines, ...

\end{block}

\vspace{15em}

\end{frame}

%%%%%%%%%%%%%%%%%%%%%%%%%%%%%%%%%%%%%%%%%%

\begin{frame}[fragile]{Visual cues: Color  (categorical or numerical)}

\begin{block}{Belonging to which group? To what extent? How severely?}

e.g. points, lines, tiles ...

\end{block}

\vspace{15em}

\end{frame}

%%%%%%%%%%%%%%%%%%%%%%%%%%%%%%%%%%%%%%%%%%

%%%%%%%%%%%%%%%%%%%%%%%%%%%%%%%%%%%%%%%%%%

\begin{frame}[fragile]{Research on perception of cues}

\begin{block}{In 1980s, \href{https://courses.ischool.berkeley.edu/i247/f05/readings/Cleveland_GraphicalPerception_Science85.pdf}{Cleveland and McGill} ran experiments to measure accuracy of human perception based on different visual cues.}

\includegraphics[width=.45\textwidth]{figure-static/perception.jpeg}

\tiny (Figure credit: \href{https://paldhous.github.io/ucb/2016/dataviz/week2.html}{Peter Aldhous})
\end{block}

\vspace{15em}

\end{frame}

% \begin{frame}[fragile]{Visualization excellence}
%
% In Tufte's words:
%
% \begin{columns}
% \begin{column}{0.5\textwidth}
% \begin{itemize}
%      \item consists of complex ideas communicated with clarity, precision, and efficiency.
%      \item is that which gives to the viewer the greatest number of ideas in the shortest time with the least ink in the smallest space.
%      \item is nearly always multivariate.
%      \item requires telling the truth about the data.
% \end{itemize}
% \end{column}
% \begin{column}{0.5\textwidth}
% \includegraphics[width=\textwidth]{figure-static/tufte-cover.jpg}
% \end{column}
% \end{columns}
% \end{frame}

%%%%%%%%%%%%%%%%%%%%%%%%%%%%%%%%%%%%%%%%%%

\begin{frame}[fragile]{Breakout rooms}

\Large
For each of the following graphics, work in your breakout rooms to complete the note-catcher assignment.

\end{frame}


%%%%%%%%%%%%%%%%%%%%%%%%%%%%%%%%%%%%%%%%%%

% \begin{frame}[fragile]{``Cities, traffic and CO2''\footnote{from \href{http://www.pnas.org/content/112/16/4999.full.pdf}{``Cities, traffic, and CO2: A multidecadal assessment of trends, drivers, and scaling relationships'', Gately et al, PNAS, 2015.} Also on Moodle.}}
%
% \includegraphics[width=\textwidth]{figure-static/pollution.png}
%
% \end{frame}
%

%%%%%%%%%%%%%%%%%%%%%%%%%%%%%%%%%%%%%%%%%%

\begin{frame}[fragile]{Front-runners in The ‘Bachelor’ Or ‘Bachelorette’\footnote{from \href{https://fivethirtyeight.com/features/the-bachelorette/}{FiveThirtyEight}}}

\includegraphics[width=\textwidth]{figure-static/koeze-hickey-bachelor-5-edited.png}

\end{frame}

%%%%%%%%%%%%%%%%%%%%%%%%%%%%%%%%%%%%%%%%%%

\begin{frame}[fragile]{``Ageing on Facebook''}

\includegraphics[width=\textwidth]{figure-static/graphic-detail-ageing-facebook.png}

\end{frame}



\end{document}
