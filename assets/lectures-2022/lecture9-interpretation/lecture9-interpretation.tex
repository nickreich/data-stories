%% beamer/knitr slides
%% for Statistical Modeling and Data Visualization course @ UMass
%% Nicholas Reich: nick [at] schoolph.umass.edu


\documentclass[table]{beamer}\usepackage[]{graphicx}\usepackage[]{color}
% maxwidth is the original width if it is less than linewidth
% otherwise use linewidth (to make sure the graphics do not exceed the margin)
\makeatletter
\def\maxwidth{ %
  \ifdim\Gin@nat@width>\linewidth
    \linewidth
  \else
    \Gin@nat@width
  \fi
}
\makeatother

\definecolor{fgcolor}{rgb}{0.345, 0.345, 0.345}
\newcommand{\hlnum}[1]{\textcolor[rgb]{0.686,0.059,0.569}{#1}}%
\newcommand{\hlstr}[1]{\textcolor[rgb]{0.192,0.494,0.8}{#1}}%
\newcommand{\hlcom}[1]{\textcolor[rgb]{0.678,0.584,0.686}{\textit{#1}}}%
\newcommand{\hlopt}[1]{\textcolor[rgb]{0,0,0}{#1}}%
\newcommand{\hlstd}[1]{\textcolor[rgb]{0.345,0.345,0.345}{#1}}%
\newcommand{\hlkwa}[1]{\textcolor[rgb]{0.161,0.373,0.58}{\textbf{#1}}}%
\newcommand{\hlkwb}[1]{\textcolor[rgb]{0.69,0.353,0.396}{#1}}%
\newcommand{\hlkwc}[1]{\textcolor[rgb]{0.333,0.667,0.333}{#1}}%
\newcommand{\hlkwd}[1]{\textcolor[rgb]{0.737,0.353,0.396}{\textbf{#1}}}%
\let\hlipl\hlkwb

\usepackage{framed}
\makeatletter
\newenvironment{kframe}{%
 \def\at@end@of@kframe{}%
 \ifinner\ifhmode%
  \def\at@end@of@kframe{\end{minipage}}%
  \begin{minipage}{\columnwidth}%
 \fi\fi%
 \def\FrameCommand##1{\hskip\@totalleftmargin \hskip-\fboxsep
 \colorbox{shadecolor}{##1}\hskip-\fboxsep
     % There is no \\@totalrightmargin, so:
     \hskip-\linewidth \hskip-\@totalleftmargin \hskip\columnwidth}%
 \MakeFramed {\advance\hsize-\width
   \@totalleftmargin\z@ \linewidth\hsize
   \@setminipage}}%
 {\par\unskip\endMakeFramed%
 \at@end@of@kframe}
\makeatother

\definecolor{shadecolor}{rgb}{.97, .97, .97}
\definecolor{messagecolor}{rgb}{0, 0, 0}
\definecolor{warningcolor}{rgb}{1, 0, 1}
\definecolor{errorcolor}{rgb}{1, 0, 0}
\newenvironment{knitrout}{}{} % an empty environment to be redefined in TeX

\usepackage{alltt}


\input{../../slide-includes/standard-knitr-beamer-preamble}

%        The following variables are assumed by the standard preamble:
%        Global variable containing module name:

\title{Fitting and interpreting model coefficients}
%	Global variable containing module shortname:
%		(Currently unused, may be used in future.)
\newcommand{\ModuleShortname}{modeling}
%	Global variable containing author name:
\author{Nicholas G Reich}
%	Global variable containing text of license terms:
\newcommand{\LicenseText}{Made available under the Creative Commons Attribution-ShareAlike 3.0 Unported License: http://creativecommons.org/licenses/by-sa/3.0/deed.en\textunderscore US }
%	Instructor: optional, can leave blank.
%		Recommended format: {Instructor: Jane Doe}
\newcommand{\Instructor}{}
%	Course: optional, can leave blank.
%		Recommended format: {Course: Biostatistics 101}
\newcommand{\Course}{}


\input{../../slide-includes/shortcuts}
\usepackage{bbm}
\usepackage{soul}

\hypersetup{colorlinks,linkcolor=,urlcolor=MainColor}


%	******	Document body begins here	**********************
\IfFileExists{upquote.sty}{\usepackage{upquote}}{}
\begin{document}

%	Title page
\begin{frame}[plain]
	\titlepage
\end{frame}

%	******	Everything through the above line must be placed at
%		the top of any TeX file using the statsTeachR standard
%		beamer preamble.





%%%%%%%%%%%%%%%%%%%%%%%%%%%%%%%%%%%%%%%%%%


%%%%%%%%%%%%%%%%%%%%%%%%%%%%%%%%%%%%%%%%%%



\begin{frame}{Today's topics}

\bi
    \myitem Model terms: recap
    \myitem Fitting and interpreting models
\ei



\end{frame}

%%%%%%%%%%%%%%%%%%%%%%%%%%%%%%%%%%%%%%%%%%

\begin{frame}[fragile]{Model terms: recap}

\bi
  \myitem {\bf The intercept} is a ``baseline`` that is included in nearly every model. What would your guess of disease severity be in the absence of any other information?
  \myitem {\bf Main terms} model the effect of explanatory variables directly.
  \myitem {\bf Interaction terms} allow for different explanatory variables to modulate the relationship of each other to the response variable.
\ei

\end{frame}

%%%%%%%%%%%%%%%%%%%%%%%%%%%%%%%%%%%%%%%%%%

\begin{frame}[fragile]{Formulas for Statistical Models (Linear Regression)}

In general, linear models can be thought of as having these components
\begin{eqnarray*}
\mbox{ y } & = & \mbox{intercept + terms + error} \\
\end{eqnarray*}


With a single predictor variable, this is simply a line (plus error):
\begin{eqnarray*}
y_i & = & \beta_0 + \beta_1 \cdot x_i + \epsilon_i
\end{eqnarray*}

However, there can be multiple variables and different types of ``terms'' in this equation
\begin{itemize}
    \item intercept
    \item main effects
    \item interaction terms
    \item smooth terms
\end{itemize}

\end{frame}



%%%%%%%%%%%%%%%%%%%%%%%%%%%%%%%%%%%%%%%%%%

\begin{frame}[fragile]{Main effects model terms}

$$ \mbox{equation: \ }  \widehat{disease}_i = \beta_0 + \beta_1\cdot crowding_i $$

\begin{knitrout}\scriptsize
\definecolor{shadecolor}{rgb}{0.969, 0.969, 0.969}\color{fgcolor}\begin{kframe}
\begin{alltt}
\hlstd{m1} \hlkwb{<-} \hlkwd{lm}\hlstd{(disease} \hlopt{~} \hlstd{crowding,} \hlkwc{data}\hlstd{=dat)}
\end{alltt}
\end{kframe}
\end{knitrout}

\begin{knitrout}\scriptsize
\definecolor{shadecolor}{rgb}{0.969, 0.969, 0.969}\color{fgcolor}
\includegraphics[width=\maxwidth]{figure/unnamed-chunk-2-1} 
\end{knitrout}


\end{frame}


%%%%%%%%%%%%%%%%%%%%%%%%%%%%%%%%%%%%%%%%%%

\begin{frame}[fragile]{Dust off your algebra: what is an intercept?}

$$ y = \beta_1 \cdot x \mbox{\ \ \ \  vs. \ \ \ \ } y = \beta_0 + \beta_1 \cdot x $$

\vspace{12em}


\end{frame}

%%%%%%%%%%%%%%%%%%%%%%%%%%%%%%%%%%%%%%%%%%

\begin{frame}[fragile]{Main effects with no intercept: bad idea}

$$ \mbox{equation: \ }  \widehat{disease}_i = \beta_1\cdot crowding_i $$

\begin{knitrout}\scriptsize
\definecolor{shadecolor}{rgb}{0.969, 0.969, 0.969}\color{fgcolor}\begin{kframe}
\begin{alltt}
\hlstd{m1_no_intcpt} \hlkwb{<-} \hlkwd{lm}\hlstd{(disease} \hlopt{~} \hlstd{crowding} \hlopt{-} \hlnum{1}\hlstd{,} \hlkwc{data}\hlstd{=dat)}
\end{alltt}
\end{kframe}
\end{knitrout}


\begin{knitrout}\scriptsize
\definecolor{shadecolor}{rgb}{0.969, 0.969, 0.969}\color{fgcolor}
\includegraphics[width=\maxwidth]{figure/unnamed-chunk-4-1} 
\end{knitrout}

\end{frame}


%%%%%%%%%%%%%%%%%%%%%%%%%%%%%%%%%%%%%%%%%%


\begin{frame}[fragile]{Main effects model terms}

$$ \mbox{equation: \ }  \widehat{disease}_i = \beta_0 + \beta_1\cdot crowding_i $$

\begin{knitrout}\scriptsize
\definecolor{shadecolor}{rgb}{0.969, 0.969, 0.969}\color{fgcolor}\begin{kframe}
\begin{alltt}
\hlstd{m1} \hlkwb{<-} \hlkwd{lm}\hlstd{(disease} \hlopt{~} \hlstd{crowding,} \hlkwc{data}\hlstd{=dat)}
\hlkwd{coef}\hlstd{(m1)}
\end{alltt}
\begin{verbatim}
## (Intercept)    crowding 
##   12.991536    1.508806
\end{verbatim}
\end{kframe}
\end{knitrout}

\begin{knitrout}\scriptsize
\definecolor{shadecolor}{rgb}{0.969, 0.969, 0.969}\color{fgcolor}
\includegraphics[width=\maxwidth]{figure/unnamed-chunk-6-1} 
\end{knitrout}


\end{frame}


%%%%%%%%%%%%%%%%%%%%%%%%%%%%%%%%%%%%%%%%%%


\begin{frame}[fragile]{Main effects model terms: interpretation}

$$ \mbox{equation: \ }  \widehat{disease}_i = \beta_0 + \beta_1\cdot crowding_i $$

\begin{knitrout}\scriptsize
\definecolor{shadecolor}{rgb}{0.969, 0.969, 0.969}\color{fgcolor}\begin{kframe}
\begin{alltt}
\hlstd{m1} \hlkwb{<-} \hlkwd{lm}\hlstd{(disease} \hlopt{~} \hlstd{crowding,} \hlkwc{data}\hlstd{=dat)}
\hlkwd{coef}\hlstd{(m1)}
\end{alltt}
\begin{verbatim}
## (Intercept)    crowding 
##   12.991536    1.508806
\end{verbatim}
\end{kframe}
\end{knitrout}



\end{frame}


%%%%%%%%%%%%%%%%%%%%%%%%%%%%%%%%%%%%%%%%%%


\begin{frame}[fragile]{Main effects model terms: interpretation}

$$ \mbox{equation: \ }  \widehat{disease}_i = \beta_0 + \beta_1\cdot crowding_i $$

\begin{knitrout}\scriptsize
\definecolor{shadecolor}{rgb}{0.969, 0.969, 0.969}\color{fgcolor}\begin{kframe}
\begin{alltt}
\hlstd{m1} \hlkwb{<-} \hlkwd{lm}\hlstd{(disease} \hlopt{~} \hlstd{crowding,} \hlkwc{data}\hlstd{=dat)}
\hlkwd{coef}\hlstd{(m1)}
\end{alltt}
\begin{verbatim}
## (Intercept)    crowding 
##   12.991536    1.508806
\end{verbatim}
\end{kframe}
\end{knitrout}

$\beta_0 $ is the expected value of $disease$ when $crowding=0$.


\end{frame}


%%%%%%%%%%%%%%%%%%%%%%%%%%%%%%%%%%%%%%%%%%


\begin{frame}[fragile]{Main effects model terms: interpretation}

$$ \mbox{equation: \ }  \widehat{disease}_i = \beta_0 + \beta_1\cdot crowding*_i $$

\begin{knitrout}\scriptsize
\definecolor{shadecolor}{rgb}{0.969, 0.969, 0.969}\color{fgcolor}\begin{kframe}
\begin{alltt}
\hlstd{dat}\hlopt{$}\hlstd{crowding_ctr} \hlkwb{<-} \hlstd{dat}\hlopt{$}\hlstd{crowding} \hlopt{-} \hlkwd{mean}\hlstd{(dat}\hlopt{$}\hlstd{crowding)}
\hlstd{m1a} \hlkwb{<-} \hlkwd{lm}\hlstd{(disease} \hlopt{~} \hlstd{crowding_ctr,} \hlkwc{data}\hlstd{=dat)}
\hlkwd{coef}\hlstd{(m1a)}
\end{alltt}
\begin{verbatim}
##  (Intercept) crowding_ctr 
##    49.919192     1.508806
\end{verbatim}
\end{kframe}
\end{knitrout}

$\beta_0 $ is the expected value of $disease$ when $crowding_{ctr}=0$, in other words, when crowding is the average value.


\end{frame}


%%%%%%%%%%%%%%%%%%%%%%%%%%%%%%%%%%%%%%%%%%


\begin{frame}[fragile]{Main effects model terms: interpretation}

$$ \mbox{equation: \ }  \widehat{disease}_i = \beta_0 + \beta_1\cdot crowding_i $$

\begin{knitrout}\scriptsize
\definecolor{shadecolor}{rgb}{0.969, 0.969, 0.969}\color{fgcolor}\begin{kframe}
\begin{alltt}
\hlstd{m1} \hlkwb{<-} \hlkwd{lm}\hlstd{(disease} \hlopt{~} \hlstd{crowding,} \hlkwc{data}\hlstd{=dat)}
\hlkwd{coef}\hlstd{(m1)}
\end{alltt}
\begin{verbatim}
## (Intercept)    crowding 
##   12.991536    1.508806
\end{verbatim}
\end{kframe}
\end{knitrout}

$\beta_1$ is the expected change in disease for a 1 unit increase of crowding.


\end{frame}


%%%%%%%%%%%%%%%%%%%%%%%%%%%%%%%%%%%%%%%%%%

\begin{frame}[fragile]{2 main effects: 1 continous, 1 categorical}

$$ \mbox{equation: \ }  \widehat{disease}_i = \beta_0 + \beta_1\cdot crowding_i + \beta_2 \cdot smoker $$

\begin{knitrout}\scriptsize
\definecolor{shadecolor}{rgb}{0.969, 0.969, 0.969}\color{fgcolor}\begin{kframe}
\begin{alltt}
\hlstd{m2} \hlkwb{<-} \hlkwd{lm}\hlstd{(disease} \hlopt{~} \hlstd{crowding} \hlopt{+} \hlstd{smoking,} \hlkwc{data}\hlstd{=dat)}
\end{alltt}
\end{kframe}
\end{knitrout}


\begin{knitrout}\scriptsize
\definecolor{shadecolor}{rgb}{0.969, 0.969, 0.969}\color{fgcolor}
\includegraphics[width=\maxwidth]{figure/unnamed-chunk-12-1} 
\end{knitrout}

\end{frame}


%%%%%%%%%%%%%%%%%%%%%%%%%%%%%%%%%%%%%%%%%%

\begin{frame}[fragile]{2 main effects: 1 continous, 1 categorical}

$$ \mbox{equation: \ }  \widehat{disease}_i = \beta_0 + \beta_1\cdot crowding_i + \beta_2 \cdot smoker $$

\begin{knitrout}\scriptsize
\definecolor{shadecolor}{rgb}{0.969, 0.969, 0.969}\color{fgcolor}\begin{kframe}
\begin{alltt}
\hlstd{m2} \hlkwb{<-} \hlkwd{lm}\hlstd{(disease} \hlopt{~} \hlstd{crowding} \hlopt{+} \hlstd{smoking,} \hlkwc{data}\hlstd{=dat)}
\hlkwd{coef}\hlstd{(m2)}
\end{alltt}
\begin{verbatim}
##   (Intercept)      crowding smokingsmoker 
##    24.0079027     0.8302413    10.4442068
\end{verbatim}
\end{kframe}
\end{knitrout}

\end{frame}


%%%%%%%%%%%%%%%%%%%%%%%%%%%%%%%%%%%%%%%%%%

\begin{frame}[fragile]{2 main effects: 1 continous, 1 categorical}

$$ \mbox{equation: \ }  \widehat{disease}_i = \beta_0 + \beta_1\cdot crowding_i + \beta_2 \cdot smoker $$

\begin{knitrout}\scriptsize
\definecolor{shadecolor}{rgb}{0.969, 0.969, 0.969}\color{fgcolor}\begin{kframe}
\begin{alltt}
\hlstd{m2} \hlkwb{<-} \hlkwd{lm}\hlstd{(disease} \hlopt{~} \hlstd{crowding} \hlopt{+} \hlstd{smoking,} \hlkwc{data}\hlstd{=dat)}
\hlkwd{coef}\hlstd{(m2)}
\end{alltt}
\begin{verbatim}
##   (Intercept)      crowding smokingsmoker 
##    24.0079027     0.8302413    10.4442068
\end{verbatim}
\end{kframe}
\end{knitrout}

$\beta_0$ is the expected value of disease when both crowding and smoker are zero.

$\beta_1$ is the expected change in disease for a one-unit change in crowding, holding smoking status constant.

$\beta_2$ is the expected difference in disease between smokers and non-smokers, holding crowding constant.


\end{frame}



%%%%%%%%%%%%%%%%%%%%%%%%%%%%%%%%%%%%%%%%%%

\begin{frame}[fragile]{Interaction model terms}

$$ \mbox{equation: \ }  \widehat{disease}_i = \beta_0 + \beta_1\cdot crowd_i + \beta_2\cdot smoke_i + \beta_3\cdot crowd_i \cdot smoke_i  $$

\begin{knitrout}\scriptsize
\definecolor{shadecolor}{rgb}{0.969, 0.969, 0.969}\color{fgcolor}\begin{kframe}
\begin{alltt}
\hlstd{m3} \hlkwb{<-} \hlkwd{coef}\hlstd{(}\hlkwd{lm}\hlstd{(disease} \hlopt{~} \hlstd{crowding}\hlopt{*}\hlstd{smoking,} \hlkwc{data}\hlstd{=dat))}
\end{alltt}
\end{kframe}
\end{knitrout}


\begin{knitrout}\scriptsize
\definecolor{shadecolor}{rgb}{0.969, 0.969, 0.969}\color{fgcolor}
\includegraphics[width=\maxwidth]{figure/interaction-term-1} 
\end{knitrout}

\end{frame}


%%%%%%%%%%%%%%%%%%%%%%%%%%%%%%%%%%%%%%%%%%
\begin{frame}{Interaction vs. confounding}

\begin{block}{Definition of interaction}
Interaction occurs when the relationship between two variables depends on the value of a third variable. E.g. you could hypothesize that the true relationship between nutritional intake and disease severity may be different for smokers and non-smokers.
\end{block}

\begin{block}{Definition of confounding}
Confounding occurs when the measurable association between two variables is distorted by the presence of another variable. Confounding can lead to biased estimates of a true relationship between variables.
\end{block}

\bi
    \myitem It is important to include confounding variables. Not doing so may bias your results.
    \myitem Unmodeled interactions do not lead to ``biased'' estimates in the same way that confounding does, but it can lead to a richer and more detailed description of the data at hand.
\ei

%[Good overview: KNN pp. 306--313]

\end{frame}

%
% %%%%%%%%%%%%%%%%%%%%%%%%%%%%%%%%%%%%%%%%%%
% \begin{frame}{Some real world examples?}
%
% \end{frame}
%

%%%%%%%%%%%%%%%%%%%%%%%%%%%%%%%%%%%%%%%%%%
\begin{frame}{How to include interaction in a MLR}


Model A: $ y_i = \beta_0 + \beta_1 x_{1i} + \beta_2 x_{2i} + \epsilon_i$


Model B: $ y_i = \beta_0 + \beta_1 x_{1i} + \beta_2 x_{2i} + \beta_3 x_{1i}\cdot x_{2i} + \epsilon_i$

\vspace{4em}

\begin{block}{Key points}
\bi
        \myitem ``easily'' conceptualized with 1 continuous, 1 categorical variable
        \myitem models possible with other variable combinations, but interpretation/visualization harder
        \myitem two variable interactions are considered ``first-order'' interactions
        \myitem still a {\bf linear} model, but no longer a strictly {\bf additive} model
\ei
\end{block}

\end{frame}


%%%%%%%%%%%%%%%%%%%%%%%%%%%%%%%%%%%%%%%%%%

\begin{frame}{How to interpret an interaction model}

For now, assume $x_1$ is continuous, $x_2$ is 0/1 binary.

Model A: $ y_i = \beta_0 + \beta_1 x_{1i} + \beta_2 x_{2i} + \epsilon_i$

Model B: $ y_i = \beta_0 + \beta_1 x_{1i} + \beta_2 x_{2i} + \beta_3 x_{1i}\cdot x_{2i} + \epsilon_i$

\vspace{12em}

\end{frame}


%%%%%%%%%%%%%%%%%%%%%%%%%%%%%%%%%%%%%%%%%%

\begin{frame}{How to interpret an interaction model}

For now, assume $x_1$ is continuous, $x_2$ is 0/1 binary.

Model A: $ y_i = \beta_0 + \beta_1 x_{1i} + \beta_2 x_{2i} + \epsilon_i$

Model B: $ y_i = \beta_0 + \beta_1 x_{1i} + \beta_2 x_{2i} + \beta_3 x_{1i}\cdot x_{2i} + \epsilon_i$

\vspace{1em}

$\beta_3$ is the change in the slope of the line that describes the relationship of $y \sim x_1$ comparing the groups defined by $x_2=0$ and $x_2=1$.

$\beta_1 + \beta_3$ is the expected change in $y$ for a one-unit increase in $x_1$ in the group $x_2=1$.

$\beta_0 + \beta_2$ is the expected value of $y$ in the group $x_2=1$ when $x_1=0$ .


\end{frame}


%%%%%%%%%%%%%%%%%%%%%%%%%%%%%%%%%%%%%%%%%%

\begin{frame}[fragile]{Fitting models in R: syntax summary}

\begin{block}{Quick recap of key syntax for linear models}
\bi
  \myitem For linear models, use {\tt lm()}.
  \myitem Equations look like {\tt y $\sim$ x1 + x2}.
  \myitem Plus signs ({\tt +}) indicate main effect terms.
  \myitem Multiplication signs ({\tt *}) indicate main effect AND interaction terms.
\ei
\end{block}

\end{frame}





%%%%%%%%%%%%%%%%%%%%%%%%%%%%%%%%%%%%%%%%%%

\begin{frame}[fragile]{Group work}


\end{frame}





\end{document}
