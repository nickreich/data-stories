%% beamer/knitr slides 
%% for Statistical Modeling and Data Visualization course @ UMass
%% Nicholas Reich: nick [at] schoolph.umass.edu


\documentclass[table]{beamer}\usepackage[]{graphicx}\usepackage[]{color}
% maxwidth is the original width if it is less than linewidth
% otherwise use linewidth (to make sure the graphics do not exceed the margin)
\makeatletter
\def\maxwidth{ %
  \ifdim\Gin@nat@width>\linewidth
    \linewidth
  \else
    \Gin@nat@width
  \fi
}
\makeatother

\definecolor{fgcolor}{rgb}{0.345, 0.345, 0.345}
\newcommand{\hlnum}[1]{\textcolor[rgb]{0.686,0.059,0.569}{#1}}%
\newcommand{\hlstr}[1]{\textcolor[rgb]{0.192,0.494,0.8}{#1}}%
\newcommand{\hlcom}[1]{\textcolor[rgb]{0.678,0.584,0.686}{\textit{#1}}}%
\newcommand{\hlopt}[1]{\textcolor[rgb]{0,0,0}{#1}}%
\newcommand{\hlstd}[1]{\textcolor[rgb]{0.345,0.345,0.345}{#1}}%
\newcommand{\hlkwa}[1]{\textcolor[rgb]{0.161,0.373,0.58}{\textbf{#1}}}%
\newcommand{\hlkwb}[1]{\textcolor[rgb]{0.69,0.353,0.396}{#1}}%
\newcommand{\hlkwc}[1]{\textcolor[rgb]{0.333,0.667,0.333}{#1}}%
\newcommand{\hlkwd}[1]{\textcolor[rgb]{0.737,0.353,0.396}{\textbf{#1}}}%
\let\hlipl\hlkwb

\usepackage{framed}
\makeatletter
\newenvironment{kframe}{%
 \def\at@end@of@kframe{}%
 \ifinner\ifhmode%
  \def\at@end@of@kframe{\end{minipage}}%
  \begin{minipage}{\columnwidth}%
 \fi\fi%
 \def\FrameCommand##1{\hskip\@totalleftmargin \hskip-\fboxsep
 \colorbox{shadecolor}{##1}\hskip-\fboxsep
     % There is no \\@totalrightmargin, so:
     \hskip-\linewidth \hskip-\@totalleftmargin \hskip\columnwidth}%
 \MakeFramed {\advance\hsize-\width
   \@totalleftmargin\z@ \linewidth\hsize
   \@setminipage}}%
 {\par\unskip\endMakeFramed%
 \at@end@of@kframe}
\makeatother

\definecolor{shadecolor}{rgb}{.97, .97, .97}
\definecolor{messagecolor}{rgb}{0, 0, 0}
\definecolor{warningcolor}{rgb}{1, 0, 1}
\definecolor{errorcolor}{rgb}{1, 0, 0}
\newenvironment{knitrout}{}{} % an empty environment to be redefined in TeX

\usepackage{alltt}


\input{../../slide-includes/standard-knitr-beamer-preamble}

%	The following variables are assumed by the standard preamble:
%	Global variable containing module name:
\title{Conceptual introduction to \tt ggplot}
%	Global variable containing module shortname:
%		(Currently unused, may be used in future.)
\newcommand{\ModuleShortname}{introRegression}
%	Global variable containing author name:
\author{Nicholas G Reich}
%	Global variable containing text of license terms:
\newcommand{\LicenseText}{Made available under the Creative Commons Attribution-ShareAlike 3.0 Unported License: http://creativecommons.org/licenses/by-sa/3.0/deed.en\textunderscore US }
%	Instructor: optional, can leave blank.
%		Recommended format: {Instructor: Jane Doe}
\newcommand{\Instructor}{}
%	Course: optional, can leave blank.
%		Recommended format: {Course: Biostatistics 101}
\newcommand{\Course}{}


\input{../../slide-includes/shortcuts}

\hypersetup{colorlinks,linkcolor=,urlcolor=MainColor}


%	******	Document body begins here	**********************
\IfFileExists{upquote.sty}{\usepackage{upquote}}{}
\begin{document}

%	Title page
\begin{frame}[plain]
	\titlepage
\end{frame}

%	******	Everything through the above line must be placed at
%		the top of any TeX file using the statsTeachR standard
%		beamer preamble. 



%%%%%%%%%%%%%%%%%%%%%%%%%%%%%%%%%%%%%%%%%%

\begin{frame}{Choices for R graphics}

You have three central choices for making graphics in R:

\begin{itemize}
    \item ``Base R graphics''
    \item {\tt ggplot2}
    \item lattice
\end{itemize}

\vspace{2em}

I use {\tt ggplot} because:
\begin{enumerate}
    \item it is integrated with the {\tt tidyverse}
    \item it is actively developed/maintained
    \item there are a ton of extensions (see more later)
\end{enumerate}


\end{frame}


%%%%%%%%%%%%%%%%%%%%%%%%%%%%%%%%%%%%%%%%%%

\begin{frame}[fragile]{Understanding the ``grammar'' of ggplot2}

The grammar ...

\begin{itemize}
    \item layer (a `geom', a `stat', an `annotation') 
    \item aesthetics (`aes`)
    \item scales
    \item facets
    \item data
    \item ... and more here: \href{http://ggplot2.tidyverse.org/reference/}{http://ggplot2.tidyverse.org/reference/}
\end{itemize}

\end{frame}

%%%%%%%%%%%%%%%%%%%%%%%%%%%%%%%%%%%%%%%%%%

\begin{frame}[fragile]{What is a layer?}


\begin{block}{Layers define the basic structure of the elements on the plot}
\begin{itemize}
    \item \href{https://ggplot2.tidyverse.org/reference/#section-geoms}{Geoms}: point, line, tile, boxplot, ribbon, ...
    \item \href{https://ggplot2.tidyverse.org/reference/#section-stats}{Stats}: histogram, smooth, density, ...
    \item \href{https://ggplot2.tidyverse.org/reference/#section-annotations}{Annotation}: hline, vline, text, ...
\end{itemize}

For more info check out the documentation: \href{http://ggplot2.tidyverse.org/reference/}{http://ggplot2.tidyverse.org/reference/}.

\end{block}

\end{frame}

%%%%%%%%%%%%%%%%%%%%%%%%%%%%%%%%%%%%%%%%%%

\begin{frame}[fragile]{What are ``aesthetics''?}

Aesthetics define a mapping between {\bf tidy data} and the information required to create a specific graphic\footnote{Figure credits: Hadley Wickham}

\includegraphics[width=\textwidth]{figure-static/aes.png}

\end{frame}


%%%%%%%%%%%%%%%%%%%%%%%%%%%%%%%%%%%%%%%%%%

\begin{frame}[fragile]{\tt{geom\_point}}


Each geom has a different set of aesthetics.

What information do we need to draw a scatterplot? \\
Or, asked another way, what aesthetics do we need for {\tt geom\_point}?

\uncover<2>{
\begin{itemize}
    \item x (required)
    \item y (required)
    \item alpha
    \item color
    \item fill
    \item shape
    \item size
\end{itemize}
}
\end{frame}

%%%%%%%%%%%%%%%%%%%%%%%%%%%%%%%%%%%%%%%%%%

\begin{frame}[fragile]{\tt{geom\_line}}

What information do we need to draw a line of connected points? \\
Or, asked another way, what aesthetics do we need for {\tt geom\_line}?

\uncover<2>{
\begin{itemize}
    \item x (required)
    \item y (required)
    \item alpha
    \item color
    \item linetype
    \item size
\end{itemize}
}
\end{frame}

%%%%%%%%%%%%%%%%%%%%%%%%%%%%%%%%%%%%%%%%%%


\begin{frame}[fragile]{}

\centering
\Large
{\tt ggplot} extensions that I used in \href{https://www.medrxiv.org/content/10.1101/2021.02.03.21250974v1}{a recent paper}

\end{frame}


\begin{frame}[fragile]{{\tt gridExtra} or {\tt cowplot} for multi-plot alignment}

\includegraphics[width=.5\textwidth]{figure-static/data-and-forecast}

\end{frame}


\begin{frame}[fragile]{\href{https://wilkelab.org/ggridges/}{{\tt ggrides}} for ridgeplots}

\includegraphics[width=\textwidth]{figure-static/fig-model-ranks}

\end{frame}


\begin{frame}[fragile]{{\tt RColorBrewer} and {\tt viridis} for colors}

\includegraphics[width=.5\textwidth]{figure-static/fig-by-horizon-week}

\end{frame}


%%%%%%%%%%%%%%%%%%%%%%%%%%%%%%%%%%%%%%%%%%

\begin{frame}[fragile]{Breakout rooms}

We have made the data from the CO2 emissions figure available in Google Drive. As a group, you will be asked to complete the following tasks:

\begin{enumerate}
    \item Recreate the figure as close as possible to the original.
    \item Improve the figure. Make some changes that you think make the figure more clear.
\end{enumerate}

At the end of class, the class will vote on which figure is (1) closest to the original and (2) the best improvement.


\end{frame}





\end{document}
