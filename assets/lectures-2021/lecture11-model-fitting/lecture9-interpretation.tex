%% beamer/knitr slides
%% for Statistical Modeling and Data Visualization course @ UMass
%% Nicholas Reich: nick [at] schoolph.umass.edu


\documentclass[table]{beamer}\usepackage[]{graphicx}\usepackage[]{color}
% maxwidth is the original width if it is less than linewidth
% otherwise use linewidth (to make sure the graphics do not exceed the margin)
\makeatletter
\def\maxwidth{ %
  \ifdim\Gin@nat@width>\linewidth
    \linewidth
  \else
    \Gin@nat@width
  \fi
}
\makeatother

\definecolor{fgcolor}{rgb}{0.345, 0.345, 0.345}
\newcommand{\hlnum}[1]{\textcolor[rgb]{0.686,0.059,0.569}{#1}}%
\newcommand{\hlstr}[1]{\textcolor[rgb]{0.192,0.494,0.8}{#1}}%
\newcommand{\hlcom}[1]{\textcolor[rgb]{0.678,0.584,0.686}{\textit{#1}}}%
\newcommand{\hlopt}[1]{\textcolor[rgb]{0,0,0}{#1}}%
\newcommand{\hlstd}[1]{\textcolor[rgb]{0.345,0.345,0.345}{#1}}%
\newcommand{\hlkwa}[1]{\textcolor[rgb]{0.161,0.373,0.58}{\textbf{#1}}}%
\newcommand{\hlkwb}[1]{\textcolor[rgb]{0.69,0.353,0.396}{#1}}%
\newcommand{\hlkwc}[1]{\textcolor[rgb]{0.333,0.667,0.333}{#1}}%
\newcommand{\hlkwd}[1]{\textcolor[rgb]{0.737,0.353,0.396}{\textbf{#1}}}%
\let\hlipl\hlkwb

\usepackage{framed}
\makeatletter
\newenvironment{kframe}{%
 \def\at@end@of@kframe{}%
 \ifinner\ifhmode%
  \def\at@end@of@kframe{\end{minipage}}%
  \begin{minipage}{\columnwidth}%
 \fi\fi%
 \def\FrameCommand##1{\hskip\@totalleftmargin \hskip-\fboxsep
 \colorbox{shadecolor}{##1}\hskip-\fboxsep
     % There is no \\@totalrightmargin, so:
     \hskip-\linewidth \hskip-\@totalleftmargin \hskip\columnwidth}%
 \MakeFramed {\advance\hsize-\width
   \@totalleftmargin\z@ \linewidth\hsize
   \@setminipage}}%
 {\par\unskip\endMakeFramed%
 \at@end@of@kframe}
\makeatother

\definecolor{shadecolor}{rgb}{.97, .97, .97}
\definecolor{messagecolor}{rgb}{0, 0, 0}
\definecolor{warningcolor}{rgb}{1, 0, 1}
\definecolor{errorcolor}{rgb}{1, 0, 0}
\newenvironment{knitrout}{}{} % an empty environment to be redefined in TeX

\usepackage{alltt}


%       ************************************************
%       **        LaTeX preamble to be used with all 
%	**        statsTeachR labs/handouts.
%
%	Author: Nicholas G Reich
%	Last modified: 14 January 2014
%	************************************************

% \documentclass[table]{beamer}

%	Set theme (a nice plain one)
\usetheme{Malmoe}

%	Use named colors, set main color of theme
%		to match Web site color:
\definecolor{MainColor}{RGB}{10, 74, 109}
\colorlet{MainColorMedium}{MainColor!50}
\colorlet{MainColorLight}{MainColor!20}
\usecolortheme[named=MainColor]{structure} 

%	For tables
%[dvipsnames] [table]
\usepackage{xcolor}

%% calling tabu.sty, assuming a particular directory structure
\usepackage{../../slide-includes/tabu}	% Even fancier than tabulary
\usepackage{multirow}

%	Just for the degree symbol
\usepackage{textcomp}

%	Get rid of footline (page, author, etc. on each slide)
\setbeamertemplate{footline}{}
%	Get rid of navigation buttons
\setbeamertemplate{navigation symbols}{}

%	Make footnotes not ugly
\usepackage{hanging}
\setbeamertemplate{footnote}{\raggedright\hangpara{1em}{1}\makebox[1em][l]{\insertfootnotemark}\footnotesize\insertfootnotetext\par}

%	Text style for code snippets inline in text:
\newcommand{\codeInline}[1]{\texttt{#1}}

%	Text style for emphasis stronger than \emph:
%		(Note, this doesn't toggle the way \emph does.
%			(Note, this can be done, didn't seem worth the trouble.))
\newcommand{\strong}[1]{{\bfseries{#1}}}


%        ******	Define title page	**********************
\setbeamertemplate{title page}{
	{\color{MainColor}
	% There must be a better way than this -vspace at
	%	 the top and bottom of the page to reduce the 
	%	 bottom margin, but I can't find one that works.
	\vspace{-6em}

% 	% Go to a lot of trouble to get the title in a
% 	%	nice box, since customizing a beamer block
% 	%	does not entirely work here (I don't know why)
	\newlength{\titleBoxWidth}
	\setlength{\titleBoxWidth}{\textwidth}
	\addtolength{\titleBoxWidth}{-2.0em}
	\setlength{\fboxsep}{1.0em}
	\setlength{\fboxrule}{0pt}
	\fcolorbox{MainColor!25}{MainColor!25}{
		\parbox{\titleBoxWidth}{
			\raggedright
			\LARGE\textbf{\inserttitle}
		}	% end parbox
	}	% end fcolorbox

	\vfill
	\small{Author: \insertauthor}
	\vspace{\baselineskip}

	\small{\Course}

	\small{\Instructor}
	\vspace{\baselineskip}

	%\small{\emph{This material is part of the \strong{statsTeachR} project}}

	\vspace{0.33\baselineskip}\scriptsize{\emph{\LicenseText}}


		\vspace{-15em}

	}	% end color
	\clearpage
}	% end define title page

%        The following variables are assumed by the standard preamble:
%        Global variable containing module name:

\title{Fitting and interpreting model coefficients}
%	Global variable containing module shortname:
%		(Currently unused, may be used in future.)
\newcommand{\ModuleShortname}{modeling}
%	Global variable containing author name:
\author{Nicholas G Reich}
%	Global variable containing text of license terms:
\newcommand{\LicenseText}{Made available under the Creative Commons Attribution-ShareAlike 3.0 Unported License: http://creativecommons.org/licenses/by-sa/3.0/deed.en\textunderscore US }
%	Instructor: optional, can leave blank.
%		Recommended format: {Instructor: Jane Doe}
\newcommand{\Instructor}{}
%	Course: optional, can leave blank.
%		Recommended format: {Course: Biostatistics 101}
\newcommand{\Course}{}


\input{../../slide-includes/shortcuts}
\usepackage{bbm}
\usepackage{soul}

\hypersetup{colorlinks,linkcolor=,urlcolor=MainColor}


%	******	Document body begins here	**********************
\IfFileExists{upquote.sty}{\usepackage{upquote}}{}
\begin{document}

%	Title page
\begin{frame}[plain]
	\titlepage
\end{frame}

%	******	Everything through the above line must be placed at
%		the top of any TeX file using the statsTeachR standard
%		beamer preamble.





%%%%%%%%%%%%%%%%%%%%%%%%%%%%%%%%%%%%%%%%%%


%%%%%%%%%%%%%%%%%%%%%%%%%%%%%%%%%%%%%%%%%%



\begin{frame}{Today's topics}

\bi
    \myitem Model terms: recap
    \myitem Fitting and interpreting models
\ei



\end{frame}

%%%%%%%%%%%%%%%%%%%%%%%%%%%%%%%%%%%%%%%%%%

\begin{frame}[fragile]{Model terms: recap}

\bi
  \myitem {\bf The intercept} is a ``baseline`` that is included in nearly every model. What would your guess of disease severity be in the absence of any other information?
  \myitem {\bf Main terms} model the effect of explanatory variables directly.
  \myitem {\bf Interaction terms} allow for different explanatory variables to modulate the relationship of each other to the response variable.
\ei

\end{frame}

%%%%%%%%%%%%%%%%%%%%%%%%%%%%%%%%%%%%%%%%%%

\begin{frame}[fragile]{Formulas for Statistical Models (Linear Regression)}

In general, linear models can be thought of as having these components
\begin{eqnarray*}
\mbox{ y } & = & \mbox{intercept + terms + error} \\
\end{eqnarray*}


With a single predictor variable, this is simply a line (plus error):
\begin{eqnarray*}
y_i & = & \beta_0 + \beta_1 \cdot x_i + \epsilon_i
\end{eqnarray*}

However, there can be multiple variables and different types of ``terms'' in this equation
\begin{itemize}
    \item intercept
    \item main effects
    \item interaction terms
    \item smooth terms
\end{itemize}

\end{frame}



%%%%%%%%%%%%%%%%%%%%%%%%%%%%%%%%%%%%%%%%%%

\begin{frame}[fragile]{Main effects model terms}

$$ \mbox{equation: \ }  \widehat{disease}_i = \beta_0 + \beta_1\cdot nutrition_i $$

\begin{knitrout}\scriptsize
\definecolor{shadecolor}{rgb}{0.969, 0.969, 0.969}\color{fgcolor}\begin{kframe}
\begin{alltt}
\hlstd{m1} \hlkwb{<-} \hlkwd{lm}\hlstd{(disease} \hlopt{~} \hlstd{nutrition,} \hlkwc{data}\hlstd{=dat)}
\end{alltt}
\end{kframe}
\end{knitrout}

\begin{knitrout}\scriptsize
\definecolor{shadecolor}{rgb}{0.969, 0.969, 0.969}\color{fgcolor}\begin{kframe}
\begin{alltt}
\hlkwd{ggplot}\hlstd{(dat,} \hlkwd{aes}\hlstd{(}\hlkwc{x}\hlstd{=nutrition,} \hlkwc{y}\hlstd{=disease))} \hlopt{+}
  \hlkwd{geom_point}\hlstd{()} \hlopt{+} \hlkwd{ylim}\hlstd{(}\hlkwd{c}\hlstd{(}\hlnum{0}\hlstd{,}\hlnum{80}\hlstd{))} \hlopt{+}
  \hlkwd{geom_abline}\hlstd{(}\hlkwc{intercept} \hlstd{=} \hlkwd{coef}\hlstd{(m1)[}\hlnum{1}\hlstd{],} \hlkwc{slope} \hlstd{=} \hlkwd{coef}\hlstd{(m1)[}\hlnum{2}\hlstd{])}
\end{alltt}
\end{kframe}
\includegraphics[width=\maxwidth]{figure/unnamed-chunk-2-1} 
\end{knitrout}


\end{frame}


%%%%%%%%%%%%%%%%%%%%%%%%%%%%%%%%%%%%%%%%%%

\begin{frame}[fragile]{Dust off your algebra: what is an intercept?}

$$ y = \beta_1 \cdot x \mbox{\ \ \ \  vs. \ \ \ \ } y = \beta_0 + \beta_1 \cdot x $$

\vspace{12em}


\end{frame}

%%%%%%%%%%%%%%%%%%%%%%%%%%%%%%%%%%%%%%%%%%


\begin{frame}[fragile]{Main effects model terms}

$$ \mbox{equation: \ }  \widehat{disease}_i = \beta_0 + \beta_1\cdot nutrition_i $$

\begin{knitrout}\scriptsize
\definecolor{shadecolor}{rgb}{0.969, 0.969, 0.969}\color{fgcolor}\begin{kframe}
\begin{alltt}
\hlstd{m1} \hlkwb{<-} \hlkwd{lm}\hlstd{(disease} \hlopt{~} \hlstd{nutrition,} \hlkwc{data}\hlstd{=dat)}
\hlkwd{summary}\hlstd{(m1)}
\end{alltt}
\begin{verbatim}
## 
## Call:
## lm(formula = disease ~ nutrition, data = dat)
## 
## Residuals:
##      Min       1Q   Median       3Q      Max 
## -20.5306  -6.4197   0.3916   5.6545  26.2250 
## 
## Coefficients:
##             Estimate Std. Error t value Pr(>|t|)    
## (Intercept) 51.77502    1.78403  29.021   <2e-16 ***
## nutrition   -0.02592    0.02084  -1.244    0.216    
## ---
## Signif. codes:  
## 0 '***' 0.001 '**' 0.01 '*' 0.05 '.' 0.1 ' ' 1
## 
## Residual standard error: 9.735 on 97 degrees of freedom
## Multiple R-squared:  0.0157,	Adjusted R-squared:  0.005556 
## F-statistic: 1.548 on 1 and 97 DF,  p-value: 0.2165
\end{verbatim}
\end{kframe}
\end{knitrout}

\begin{knitrout}\scriptsize
\definecolor{shadecolor}{rgb}{0.969, 0.969, 0.969}\color{fgcolor}\begin{kframe}
\begin{alltt}
\hlkwd{ggplot}\hlstd{(dat,} \hlkwd{aes}\hlstd{(}\hlkwc{x}\hlstd{=nutrition,} \hlkwc{y}\hlstd{=disease))} \hlopt{+}
  \hlkwd{geom_point}\hlstd{()} \hlopt{+} \hlkwd{ylim}\hlstd{(}\hlkwd{c}\hlstd{(}\hlnum{0}\hlstd{,}\hlnum{80}\hlstd{))} \hlopt{+}
  \hlkwd{geom_abline}\hlstd{(}\hlkwc{intercept} \hlstd{=} \hlkwd{coef}\hlstd{(m1)[}\hlnum{1}\hlstd{],} \hlkwc{slope} \hlstd{=} \hlkwd{coef}\hlstd{(m1)[}\hlnum{2}\hlstd{])}
\end{alltt}
\end{kframe}
\includegraphics[width=\maxwidth]{figure/unnamed-chunk-4-1} 
\end{knitrout}


\end{frame}


%%%%%%%%%%%%%%%%%%%%%%%%%%%%%%%%%%%%%%%%%%

\begin{frame}[fragile]{Main effects w/no intercept}

$$ \mbox{equation: \ }  \widehat{disease}_i = \beta_1\cdot nutrition_i $$

\begin{knitrout}\scriptsize
\definecolor{shadecolor}{rgb}{0.969, 0.969, 0.969}\color{fgcolor}\begin{kframe}
\begin{alltt}
\hlstd{m1_no_intcpt} \hlkwb{<-} \hlkwd{lm}\hlstd{(disease} \hlopt{~} \hlstd{nutrition} \hlopt{-} \hlnum{1}\hlstd{,} \hlkwc{data}\hlstd{=dat)}
\hlkwd{ggplot}\hlstd{(dat,} \hlkwd{aes}\hlstd{(}\hlkwc{x}\hlstd{=nutrition,} \hlkwc{y}\hlstd{=disease))} \hlopt{+}
  \hlkwd{geom_point}\hlstd{()} \hlopt{+} \hlkwd{ylim}\hlstd{(}\hlkwd{c}\hlstd{(}\hlnum{0}\hlstd{,}\hlnum{80}\hlstd{))} \hlopt{+}
  \hlkwd{geom_abline}\hlstd{(}\hlkwc{intercept} \hlstd{=} \hlnum{0}\hlstd{,} \hlkwc{slope} \hlstd{=} \hlkwd{coef}\hlstd{(m1_no_intcpt)[}\hlnum{1}\hlstd{])}
\end{alltt}
\end{kframe}
\includegraphics[width=\maxwidth]{figure/unnamed-chunk-5-1} 
\end{knitrout}

\end{frame}

%%%%%%%%%%%%%%%%%%%%%%%%%%%%%%%%%%%%%%%%%%

\begin{frame}[fragile]{Main effects model terms (crowding)}

$$ \mbox{equation: \ }  \widehat{disease}_i = \beta_0 + \beta_1\cdot crowding_i $$

\begin{knitrout}\scriptsize
\definecolor{shadecolor}{rgb}{0.969, 0.969, 0.969}\color{fgcolor}\begin{kframe}
\begin{alltt}
\hlstd{m1} \hlkwb{<-} \hlkwd{lm}\hlstd{(disease} \hlopt{~} \hlstd{crowding,} \hlkwc{data}\hlstd{=dat)}
\hlkwd{ggplot}\hlstd{(dat,} \hlkwd{aes}\hlstd{(}\hlkwc{x}\hlstd{=crowding,} \hlkwc{y}\hlstd{=disease))} \hlopt{+} \hlkwd{geom_point}\hlstd{()} \hlopt{+}
  \hlkwd{geom_abline}\hlstd{(}\hlkwc{intercept} \hlstd{=} \hlkwd{coef}\hlstd{(m1)[}\hlnum{1}\hlstd{],} \hlkwc{slope} \hlstd{=} \hlkwd{coef}\hlstd{(m1)[}\hlnum{2}\hlstd{])}
\end{alltt}
\end{kframe}
\includegraphics[width=\maxwidth]{figure/unnamed-chunk-6-1} 
\end{knitrout}

\end{frame}


%%%%%%%%%%%%%%%%%%%%%%%%%%%%%%%%%%%%%%%%%%

\begin{frame}[fragile]{2 main effects: 1 continous, 1 categorical}

$$ \mbox{equation: \ }  \widehat{disease}_i = \beta_0 + \beta_1\cdot crowding_i + \beta_2 \cdot smoker $$

\begin{knitrout}\scriptsize
\definecolor{shadecolor}{rgb}{0.969, 0.969, 0.969}\color{fgcolor}\begin{kframe}
\begin{alltt}
\hlstd{m2} \hlkwb{<-} \hlkwd{lm}\hlstd{(disease} \hlopt{~} \hlstd{crowding} \hlopt{+} \hlstd{smoking,} \hlkwc{data}\hlstd{=dat)}
\hlkwd{ggplot}\hlstd{(dat,} \hlkwd{aes}\hlstd{(}\hlkwc{x}\hlstd{=crowding,} \hlkwc{y}\hlstd{=disease,} \hlkwc{color}\hlstd{=smoking))} \hlopt{+} \hlkwd{geom_point}\hlstd{()} \hlopt{+}
  \hlkwd{scale_color_manual}\hlstd{(}\hlkwc{values} \hlstd{=} \hlkwd{c}\hlstd{(}\hlstr{"blue"}\hlstd{,} \hlstr{"red"}\hlstd{))} \hlopt{+}
  \hlkwd{geom_abline}\hlstd{(}\hlkwc{intercept} \hlstd{=} \hlkwd{coef}\hlstd{(m2)[}\hlnum{1}\hlstd{],} \hlkwc{slope} \hlstd{=} \hlkwd{coef}\hlstd{(m2)[}\hlnum{2}\hlstd{],} \hlkwc{color}\hlstd{=}\hlstr{"blue"}\hlstd{)} \hlopt{+}
  \hlkwd{geom_abline}\hlstd{(}\hlkwc{intercept} \hlstd{=} \hlkwd{coef}\hlstd{(m2)[}\hlnum{1}\hlstd{]}\hlopt{+}\hlkwd{coef}\hlstd{(m2)[}\hlnum{3}\hlstd{],} \hlkwc{slope} \hlstd{=} \hlkwd{coef}\hlstd{(m2)[}\hlnum{2}\hlstd{],} \hlkwc{color}\hlstd{=}\hlstr{"red"}\hlstd{)}
\end{alltt}
\end{kframe}
\includegraphics[width=\maxwidth]{figure/unnamed-chunk-7-1} 
\end{knitrout}

\end{frame}


%%%%%%%%%%%%%%%%%%%%%%%%%%%%%%%%%%%%%%%%%%

\begin{frame}[fragile]{Interaction model terms}

$$ \mbox{equation: \ }  \widehat{disease}_i = \beta_0 + \beta_1\cdot crowd_i + \beta_2\cdot smoke_i + \beta_3\cdot crowd_i \cdot smoke_i  $$

\begin{knitrout}\scriptsize
\definecolor{shadecolor}{rgb}{0.969, 0.969, 0.969}\color{fgcolor}\begin{kframe}
\begin{alltt}
\hlstd{m3} \hlkwb{<-} \hlkwd{coef}\hlstd{(}\hlkwd{lm}\hlstd{(disease} \hlopt{~} \hlstd{crowding}\hlopt{*}\hlstd{smoking,} \hlkwc{data}\hlstd{=dat))}
\hlkwd{ggplot}\hlstd{(dat,} \hlkwd{aes}\hlstd{(}\hlkwc{x}\hlstd{=crowding,} \hlkwc{y}\hlstd{=disease,} \hlkwc{color}\hlstd{=smoking))} \hlopt{+}  \hlkwd{geom_point}\hlstd{()} \hlopt{+}
  \hlkwd{scale_color_manual}\hlstd{(}\hlkwc{values} \hlstd{=} \hlkwd{c}\hlstd{(}\hlstr{"blue"}\hlstd{,} \hlstr{"red"}\hlstd{))} \hlopt{+}
  \hlkwd{geom_abline}\hlstd{(}\hlkwc{intercept} \hlstd{= m3[}\hlnum{1}\hlstd{],} \hlkwc{slope} \hlstd{= m3[}\hlnum{2}\hlstd{],} \hlkwc{color}\hlstd{=}\hlstr{"blue"}\hlstd{)} \hlopt{+}
  \hlkwd{geom_abline}\hlstd{(}\hlkwc{intercept} \hlstd{= m3[}\hlnum{1}\hlstd{]}\hlopt{+}\hlstd{m3[}\hlnum{3}\hlstd{],} \hlkwc{slope} \hlstd{= m3[}\hlnum{2}\hlstd{]}\hlopt{+}\hlstd{m3[}\hlnum{4}\hlstd{],} \hlkwc{color}\hlstd{=}\hlstr{"red"}\hlstd{)}
\end{alltt}
\end{kframe}
\includegraphics[width=\maxwidth]{figure/interaction-term-1} 
\end{knitrout}

\end{frame}


%%%%%%%%%%%%%%%%%%%%%%%%%%%%%%%%%%%%%%%%%%

\begin{frame}[fragile]{Interaction via {\tt geom\_smooth()}}

$$ \mbox{equation: \ }  \widehat{disease}_i = \beta_0 + \beta_1\cdot crowd_i + \beta_2\cdot smoke_i + \beta_3\cdot crowd_i \cdot smoke_i  $$

\begin{knitrout}\scriptsize
\definecolor{shadecolor}{rgb}{0.969, 0.969, 0.969}\color{fgcolor}\begin{kframe}
\begin{alltt}
\hlkwd{ggplot}\hlstd{(dat,} \hlkwd{aes}\hlstd{(}\hlkwc{x}\hlstd{=crowding,} \hlkwc{y}\hlstd{=disease,} \hlkwc{color}\hlstd{=smoking))} \hlopt{+}  \hlkwd{geom_point}\hlstd{()} \hlopt{+}
  \hlkwd{geom_smooth}\hlstd{(}\hlkwc{method}\hlstd{=}\hlstr{"lm"}\hlstd{,} \hlkwc{se}\hlstd{=}\hlnum{FALSE}\hlstd{)} \hlopt{+} \hlkwd{scale_color_manual}\hlstd{(}\hlkwc{values} \hlstd{=} \hlkwd{c}\hlstd{(}\hlstr{"blue"}\hlstd{,} \hlstr{"red"}\hlstd{))}
\end{alltt}
\end{kframe}
\includegraphics[width=\maxwidth]{figure/unnamed-chunk-8-1} 
\end{knitrout}


\end{frame}


%%%%%%%%%%%%%%%%%%%%%%%%%%%%%%%%%%%%%%%%%%

\begin{frame}[fragile]{Smooth model terms}

$$ \mbox{equation: \ }  \widehat{disease} = \beta_0 + s(education)$$

\begin{knitrout}\scriptsize
\definecolor{shadecolor}{rgb}{0.969, 0.969, 0.969}\color{fgcolor}\begin{kframe}
\begin{alltt}
\hlkwd{library}\hlstd{(splines)}
\hlstd{m4} \hlkwb{<-} \hlkwd{lm}\hlstd{(disease} \hlopt{~} \hlkwd{ns}\hlstd{(education,} \hlkwc{knots} \hlstd{=} \hlnum{8}\hlstd{),} \hlkwc{data}\hlstd{=dat)}
\hlstd{x.new} \hlkwb{<-} \hlkwd{seq}\hlstd{(}\hlkwd{min}\hlstd{(dat}\hlopt{$}\hlstd{education),} \hlkwd{max}\hlstd{(dat}\hlopt{$}\hlstd{education),} \hlkwc{by}\hlstd{=}\hlnum{.1}\hlstd{)}
\hlstd{yhat.m4} \hlkwb{<-} \hlkwd{predict}\hlstd{(m4,} \hlkwc{newdata} \hlstd{=} \hlkwd{data.frame}\hlstd{(}\hlkwc{education}\hlstd{=x.new))}
\hlkwd{ggplot}\hlstd{()} \hlopt{+} \hlkwd{geom_point}\hlstd{(}\hlkwc{data}\hlstd{=dat,} \hlkwd{aes}\hlstd{(}\hlkwc{x}\hlstd{=education,} \hlkwc{y}\hlstd{=disease))} \hlopt{+}
    \hlkwd{geom_path}\hlstd{(}\hlkwd{aes}\hlstd{(}\hlkwc{x}\hlstd{=x.new,} \hlkwc{y}\hlstd{=yhat.m4),} \hlkwc{color}\hlstd{=}\hlstr{"red"}\hlstd{)}
\end{alltt}
\end{kframe}
\includegraphics[width=\maxwidth]{figure/smooth-terms-1} 
\end{knitrout}

\end{frame}


%%%%%%%%%%%%%%%%%%%%%%%%%%%%%%%%%%%%%%%%%%

\begin{frame}[fragile]{Fitting models in R}

\begin{block}{Quick recap of key syntax for linear models}
\bi
  \myitem For linear models, use {\tt lm()}.
  \myitem Equations look like {\tt y ~ x1 + x2}.
  \myitem Plus signs ({\tt +}) indicate main effect terms.
  \myitem Multiplication signs ({\tt *}) indicate main effect AND interaction terms.
\ei
\end{block}

\end{frame}


%%%%%%%%%%%%%%%%%%%%%%%%%%%%%%%%%%%%%%%%%%

\begin{frame}[fragile]{Reading model output}

$$ \mbox{equation: \ }  \widehat{disease}_i = \beta_0 + \beta_1\cdot crowd_i + \beta_2\cdot smoke_i + \beta_3\cdot crowd_i \cdot smoke_i  $$


\begin{knitrout}\scriptsize
\definecolor{shadecolor}{rgb}{0.969, 0.969, 0.969}\color{fgcolor}\begin{kframe}
\begin{alltt}
\hlstd{m3} \hlkwb{<-} \hlkwd{lm}\hlstd{(disease} \hlopt{~} \hlstd{crowding}\hlopt{*}\hlstd{smoking,} \hlkwc{data}\hlstd{=dat)}
\hlkwd{round}\hlstd{(}\hlkwd{summary}\hlstd{(m3)}\hlopt{$}\hlstd{coef,} \hlnum{2}\hlstd{)}
\end{alltt}
\begin{verbatim}
##                        Estimate Std. Error t value Pr(>|t|)
## (Intercept)               25.31       4.90    5.17     0.00
## crowding                   0.77       0.23    3.39     0.00
## smokingsmoker              8.11       6.93    1.17     0.24
## crowding:smokingsmoker     0.10       0.29    0.34     0.73
\end{verbatim}
\end{kframe}
\end{knitrout}



\end{frame}




%%%%%%%%%%%%%%%%%%%%%%%%%%%%%%%%%%%%%%%%%%

\begin{frame}[fragile]{Breakout rooms}

Work on, and share progress on your Lab 3 with other group-mates.

\end{frame}





\end{document}
